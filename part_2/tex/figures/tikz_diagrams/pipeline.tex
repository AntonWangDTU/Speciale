\begin{figure}[htbp]
\centering
\begin{tikzpicture}[
    node distance=0.5cm and 0.5cm,
    box/.style={rectangle, draw, thick, minimum width=2.5cm, minimum height=1cm, align=center, font=\footnotesize},
    data/.style={box, fill=blue!10},
    split/.style={box, fill=yellow!20},
    arrow/.style={->, thick, >=stealth},
]
% Raw data
\node[data] (raw) {Raw scRNA-seq\\Data};
% Split node
% Three split outputs with different colors
\node[box, fill=green!20, below left=of raw, xshift=-0.5cm] (train) {Training Set\\(70\%)};
\node[box, fill=orange!20, below=of raw] (val) {Validation Set\\(20\%)};
\node[box, fill=red!20, below right=of raw, xshift=0.5cm] (test) {Test Set\\(10\%)};
% Arrows
\draw[arrow] (raw) -- (train);
\draw[arrow] (raw) -- (val);
\draw[arrow] (raw) -- (test);


\node[box, fill=green!20, below=of train, yshift=-3cm] (vaetrain) {VAE training};


\draw[arrow] (train) -- (vaetrain);

\node[box, fill=orange!20, below=of vaetrain, yshift=-2cm] (vaeval) {VAE training};

\draw[arrow] (vaetrain) -- (vaeval);

\end{tikzpicture}
\caption{Data splitting strategy.}
\label{fig:data_split}
\end{figure}