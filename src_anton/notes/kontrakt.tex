\documentclass[12pt]{article}
\usepackage[utf8]{inputenc}
\usepackage[a4paper,margin=2.5cm]{geometry}
\usepackage{setspace}
\usepackage{parskip}
\usepackage{lmodern}
\usepackage{hyperref}

\title{Specialekontrakt}
\date{}

\begin{document}

\maketitle

\section*{Titel}
\textbf{Robust Cell Typing in Homogeneous scRNA-seq Data for Quality Control in Stem Cell Therapy}

\section*{Other Information}

Supervision will be done every three weeks, or more frequently when needed. We will prepare a few slides for each session to effectively provide updates on the progress of the project. The supervisor will give relevant feedback. We will aim to make progress between each session to ensure that the project does not stagnate.

We will compose the thesis as a collective report and arrange the thesis defense as a group examination. I am writing my thesis with Gustav Helms qbg413@alumni.ku.dk, MSc Bioinformatics at university of copenhagen.

\section*{Statement of Dissertation}

The project is conducted in collaboration with the Cell Therapy Department of Novo Nordisk. The department cultivates cells derived from stem cells and aims to improve methods for quality checking of the resulting cell populations. Ideally, the final cultivated cell product would consist only of a specific target cell type. However, in practice, the samples often contain a mixture of transcriptionally similar cell types—such as beta and alpha cells—which makes the population appear highly homogeneous.

This homogeneity gives rise to challenges, as traditional cell typing and quality control procedures rely on more distinct cellular differences.

The aim of this project is to address the problem of classifying cells within a homogeneous population using single-cell RNA sequencing (scRNA-seq) data, with the goal of improving quality control procedures. The project builds upon a pre-study conducted in Spring 2025, which explored how to select optimal reference sets for reference-based cell typing.

We will investigate both reference-based and embedding-based methods for cell typing. The methods should be able to handle the batch effects that arise from cross-batch comparisons and focus on developing techniques that are robust to the absence or inaccuracy of annotations in reference datasets.

Specifically, we will continue our study of reference-based methods, and also explore the foundation model \textit{GeneCompass} for embedding-based methods. We will perform benchmarking using more traditional cell typing methods as well.

\end{document}
